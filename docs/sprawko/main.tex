\documentclass[12pt]{article}

\usepackage[utf8]{inputenc}
\usepackage[T1]{fontenc}
\usepackage[english, polish]{babel}
\usepackage[hidelinks]{hyperref}
\usepackage{import}
\usepackage{graphicx}
\usepackage{indentfirst}
\usepackage{enumerate}
\usepackage{amsmath}

\graphicspath{{images/}}

\begin{document}

\import{./}{titlepage.tex}

\tableofcontents

\section{Zadanie pierwsze}

Na pierwszy rzut oka widać podział dokumentu na część znajdującą się 
przed \texttt{\textbackslash begin\{document\}} oraz właściwą treść. Preambuła służy do 
deklaracji paczek z których będziemy korzystać przy składaniu dokumentu,
jak i do wstępnej jego konfiguracji.

\section{Zadanie drugie}

Instrukcja \texttt{\textbackslash usepackage\{...\}} umieszczana jest w
pierwszej części dokumentu i służy do deklaracji wymaganych modułów.
Z paczkami wymienionymi w poleceniu się zapoznałem, większość z nich
została także wykorzystana w tym dokumencie.

\section{Zadanie trzecie}

\subsection{Pierwsza podsekcja}

\emph{Trochę tekstu który można wykorzystać jako przykładowy w dokumencie.}
\textbf{Pewnie wkleję go kilka razy.}
\texttt{Trochę tekstu który można wykorzystać jako przykładowy w dokumencie.}
\textit{Pewnie wkleję go kilka razy.}
\textsc{Trochę tekstu który można wykorzystać jako przykładowy w dokumencie.}
\textnormal{Pewnie wkleję go kilka razy.}

Trochę tekstu który można wykorzystać jako przykładowy w dokumencie.
Pewnie wkleję go kilka razy.
Trochę tekstu który można wykorzystać jako przykładowy w dokumencie.
Pewnie wkleję go kilka razy.
Trochę tekstu który można wykorzystać jako przykładowy w dokumencie.
Pewnie wkleję go kilka razy.

\section{Zadanie czwarte}

Zaraz w tekście wystąpią rożne rodzaje list dostępne w \LaTeX\/.
Pierwsza z nich przedstawi przedmioty z których będziemy pisać egzaminy w tym semestrze.
Zostaną one posortowane od najtrudniejszego do najłatwiejszego.

\begin{enumerate}[a)]
    \item Podstawy Ochrony Danych
    \item Bazy Danych
    \item Grafika Komputerowa i Komunikacja Człowiek-Komputer
\end{enumerate}

Druga lista przedstawi spis znanych przeze mnie języków programowania:

\begin{itemize}
    \item C
    \item C++
    \item C\#
    \item Python
    \item Java
    \item Elixir
    \item JavaScript
    \item Go
\end{itemize}

\section{Zadanie piąte}

Spytałem się Huberta o to czy jest w stanie pomóc mi z tym zadaniem, na co on odpowiedział:
,,Byczku\footnote{Byczku - Kolego} jeszcze jak''.

\section{Zadanie szóste}

Geograficzne granice Polski wraz z podziałem na województwa możemy zobaczyć na rysunku \ref{fig:granice}.

\begin{figure}[p]
    \includegraphics[width=\textwidth]{granice_polski.png}
    \centering
    \caption{Geograficzne granice Polski, stan na dzień \today}
    \label{fig:granice}
\end{figure}

Ponadto w tabeli \ref{tab:wojewodztwa} możemy zauważyć kilka województw
Polski wraz z powierzchnią jaką zajmują oraz ilością ludności zamieszkałej
na danym terenie. Tabela znajduje się na stronie \pageref{tab:wojewodztwa}.

\begin{table}[p]
    \centering
    \caption{\label{tab:wojewodztwa}Wybrane województwa wraz z ich powierzchnią i populacją}
    \begin{tabular}{ |c c c c| }
        \hline
        Województwo   & Powierzchnia [km2] & Populacja \\
        \hline\hline
        Lubelskie     & 25 122             & 2 139 726 \\
        \hline
        Mazowieckie   & 35 558             & 5 349 114 \\
        \hline
        Wielkopolskie & 29 826             & 3 475 323 \\
        \hline
    \end{tabular}
\end{table}

\section{Zadanie siódme}

Wymagane równania to kolejno:

\begin{enumerate}[a)]
    \item $\binom{n}{k}= \frac{n!}{k!\left ( n-k \right )!}$
    \item $a^{2}+b^{2}=c^{2}$
    \item $\frac{\frac{1}{2}+\frac{1}{4}}{\frac{1}{8}}\neq 1$
    \item $\sum_{i=1}^{n}i^2=\frac{n(n+1)(2n+1)}{6}$
    \item $C_{2}^{-3}+O_{0}^{2}\rightarrow C^{+4}O_{2}^{-2}+H_2^{+1}O^{-2}$
    \item $A_{m,n} = \begin{pmatrix}
                      a_{1,1} & a_{1,2} & \hdots & a_{1,n} \\
                      a_{2,1} & a_{2,2} & \hdots & a_{2,n} \\
                      \vdots  & \vdots  & \ddots & \hdots  \\
                      a_{m,1} & a_{m,2} & \hdots & a_{m,n}
                  \end{pmatrix}$
\end{enumerate}

\section {Zadanie ósme}

Aby zostać pełnoprawnym artykułem dokument ten potrzebuje jeszcze
kilku pozycji bibliograficznych. Do stworzenia tego dokumentu wykorzystano
pewien świetny artykuł \cite{article}, dobrą książkę \cite{book}
oraz wybitny tutorial online \cite{misc}.

\section{Zadanie dziewiąte}

Po wykonaniu tego zadania laboratoryjnego uważam, że \LaTeX\ jest w stanie
zdecydowanie ułatwić proces składu tekstu im bardziej obszerny on jest.
Poza trudniejszym formułowaniem tabel \LaTeX\ w każdej kwestii jest lepszy
od oprogramowania biurowego z którego do tej pory korzystałem.

\bibliographystyle{unsrt}
\bibliography{references}

\end{document}