\documentclass[12pt]{article}

\usepackage[utf8]{inputenc}
\usepackage[T1]{fontenc}
\usepackage[english, polish]{babel}
\usepackage[hidelinks]{hyperref}
\usepackage{import}
\usepackage{graphicx}
\usepackage{indentfirst}
\usepackage{enumerate}
\usepackage{amsmath}
\usepackage{tabularx}

\graphicspath{{images/}}

\begin{document}

\begin{flushright}
    \today
\end{flushright}

\begin{flushleft}
    \begin{tabular}{ll}
        Bartosz Kosmala  & 140726 \\
        Hubert Knioła    & 139644 \\
        Mateusz Kuźniak  & 139092 \\
        Jordan Kondracki & 140721
    \end{tabular}
\end{flushleft}

\vspace{0.5cm}

\begin{center}
    \textbf{Sztuczna Inteligencja - laboratorium}\\[0.3cm]
    Sprawozdanie końcowe\\[0.3cm]
    \textit{\textbf{Rogo Puzzle z wykorzystaniem paradygmatu CSP/FD}}
\end{center}

\vspace{0.5cm}

\tableofcontents

\newpage

\section{Opis zadania}

Celem projektu jest stworzenie aplikacji, która umożliwia rozwiązanie łamigłówki
\textit{Rogo}, zgodnie z paradygmatem CSP (ang. \textit{Contraint Satisfaction Problem}) w skończonej
dziedzinie (ang. \textit{Finite Domain}).

Łamigłówka polega na ustaleniu na planszy ciągłej ścieżki, która zawiera pola z liczbami. 
Ścieżek na takiej planczy można znaleźć wiele ale rozwiązaniem jest ta ścieżka, 
której suma liczb na jej polach jest największa.

\section{Założenia realizacyjne}

W tym rozdziale zawarto informację o algorytmie i technologii wykorzystanej do
rozwiązania problemu.

\subsection{Zdefiniowanie problemu oraz opis algorytmu}

Rozwiązaniem CSP jest przypisanie każdej zmiennej wartości z jej dziedziny
spełniającej wszystkie ograniczenia. Zadaniem jest znalezienie jednego
lub wszystkich rozwiązań.\footnote{,,Programowanie logiczne z użyciem ograniczeń'' \cite{cotocsp}}

Łamigłówka \textit{Rogo} polega na znalezieniu na prostokątnej planszy takiej ścieżki, która:
\begin{itemize}
    \item zaczyna się w dowolnym polu planszy,
    \item jest pętlą, to znaczy, że jej początek jest bezpośredio połączony z jej końcem,
    \item nie zawiera pól niedozwolonych,
    \item składa się z pól, które są ze sobą połączone ortogonalnie,
    \item może odwiedzić każde pole tylko raz,
    \item składa się z dokładnej liczby pól wyznaczanej przy jej definiowaniu.
\end{itemize}

\subsection{Język programowania, narzędzia informatyczne, środowisko i biblioteki}

Do implementacji aplikacji wykorzystamy język programowania ogólnego przeznaczenia
\textit{Python} oraz język \textit{MiniZinc}, który pozwala na definiowanie
problemów CSP \cite{minizinc}.

Do pisania kodu wykorzystamy narzędzie \textit{Visual Studio Code}. Do kontroli
wersji aplikacji
użyjemy systemu \textit{Git} wraz z platformą \textit{GitHub}.

W implementacji skorzystamy z biblioteki \texttt{minizinc}
dla języka \textit{Python}. Pozwala ona na korzystanie ze środowiska
\textit{MiniZinc} z poziomu interpretera. Umożliwia ona m.in. komplilację modeli
oraz rozwiązywanie ich.

\section{Podział prac}

Podział prac przy projekcie pomiędzy jego członków jest widoczny na
tablicy \ref{tab:podzialprac}.

\begin{table}[h!]
    \caption{\label{tab:podzialprac}Podział prac przy projekcie}
    \begin{tabularx}{\textwidth}{ | p{\dimexpr.28\linewidth-2\tabcolsep-1.3333\arrayrulewidth} | X | }
        \hline
        Autor            & Wkład  \\
        \hline
        Bartosz Kosmala  &
        \begin{itemize}
            \item redagowanie sprawozdań roboczych
            \item implementacja interfejsu graficznego aplikacji
            \item zapoznanie się z wymaganymi technologiami
        \end{itemize} \\
        \hline
        Hubert Knioła    &
        \begin{itemize}
            \item implementacja interfejsu graficznego aplikacji
            \item zapoznanie się z wymaganymi technologiami
            \item pomoc merytoryczna przy sprawozdaniu
        \end{itemize} \\
        \hline
        Mateusz Kuźniak  &
        \begin{itemize}
            \item implementacja modelu CSP
            \item zapoznanie się z wymaganymi technologiami
            \item pomoc merytoryczna przy sprawozdaniu
        \end{itemize} \\
        \hline
        Jordan Kondracki &
        \begin{itemize}
            \item redagowanie sprawozdań roboczych
            \item implementacja modelu CSP
            \item zapoznanie się z wymaganymi technologiami
        \end{itemize} \\
        \hline
    \end{tabularx}
\end{table}

\newpage

\section{Opis implementacji}

\dots

\subsection{Struktury danych wykorzystywane w programie}

\dots

\subsection{Funkcje oraz procedury}

\dots

\section{Użytkowanie i testowanie systemu}

\dots

\subsection{Interfejs aplikacji}

\dots

\subsection{Sterowanie aplikacją}

\dots

\subsection{Testy aplikacji}

\dots

\section{Funkcjonalności dodatkowe}

\dots

\newpage

\bibliographystyle{unsrt}
\bibliography{references}

\end{document}