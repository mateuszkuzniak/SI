\documentclass[12pt]{article}

\usepackage[utf8]{inputenc}
\usepackage[T1]{fontenc}
\usepackage[english, polish]{babel}
\usepackage[hidelinks]{hyperref}
\usepackage{import}
\usepackage{graphicx}
\usepackage{indentfirst}
\usepackage{enumerate}
\usepackage{amsmath}
\usepackage{tabularx}

\graphicspath{{images/}}

\begin{document}

\begin{flushright}
    \today
\end{flushright}

\begin{flushleft}
    \begin{tabular}{ll}
        Bartosz Kosmala  & 140726 \\
        Hubert Knioła    & 139644 \\
        Mateusz Kuźniak  & 139092 \\
        Jordan Kondracki & 140721
    \end{tabular}
\end{flushleft}

\vspace{0.5cm}

\begin{center}
    \textbf{Sztuczna Inteligencja - laboratorium}\\[0.3cm]
    Sprawozdanie robocze, wariant 2\\[0.3cm]
    \textit{\textbf{Algorytm rozwiązujący Bag Puzzle z wykorzystaniem paradygmatu CSP/FD}}
\end{center}

\vspace{0.5cm}

\section{Zmiany}

W tym rozdziale opiszemy zmiany wprowadzone w ostatnich
dwóch tygodniach.

\subsection{Aspekt wizualny - Bartosz Kosmala i Hubert Knioła}

Udało nam się przygotować aplikację okienkową, pozwalającą
na rozwiązywanie łamigłówki \textit{Bag Puzzle}. Jest ona
widoczna na rysunkach \ref{fig:okno1} oraz \ref{fig:okno2}.

Na obecną chwilę jej funkcjonalność ogranicza się do odczytu
pojedynczej planszy oraz możliwości jej wypełniania.
Działa również przycisk resetujący planszę.

\subsection{Aspekt wykorzystania CSP - Jordan Kondracki i Mateusz Kuźniak}

W ostatnim okresie skupiliśmy się na studiowaniu modelowania
problemów CSP za pomocą tutoriali oferowanych przez
\textit{MiniZinc} \cite{modelling} jak i sposobach rozwiązania
bardziej złożonych łamigłówek z artykułu 
,,\textit{Solving Challenging Grid Puzzles with Answer Set Programming}'' \cite{solving}.

Po przeanalizowaniu wszystkich materiałów doszliśmy do wniosku,
że problem nam przydzielony jest znacznie trudniejszy niż się
spodziewaliśmy. W drugim artykule przytoczone jest co prawda
rozwiązanie tego problemu w języku \texttt{LParse}, jednak 
ciężko jest zrozumieć co on reprezentuje. Dodatkowo, w internecie
bardzo trudno jest zdobyć jakiekolwiek materiały 
dotyczące samego języka \texttt{LParse}.

\bibliographystyle{unsrt}
\bibliography{references}

\begin{figure}[p]
    \includegraphics[width=0.8\textwidth]{1.png}
    \centering
    \caption{Zrzut ekranu aplikacji wizualizującej łamigłówkę}
    \label{fig:okno1}
\end{figure}

\begin{figure}[p]
    \includegraphics[width=0.8\textwidth]{2.png}
    \centering
    \caption{Zrzut ekranu aplikacji wizualizującej łamigłówkę}
    \label{fig:okno2}
\end{figure}

\end{document}